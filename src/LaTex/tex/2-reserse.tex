\chapter{Rešerše}
\label{2-reserse}
%https://dspace.cvut.cz/handle/10467/7186
Nepodařilo se dohledat žádnou práci, která by se zabývala explicitně tvorbou \text{zařízení} pro příjem a záznam koordinovaného času pro účely astronomického měření. \text{Podařilo} se však dohledat zdroje, které se zabývají souvisejícími tématy. Geodetickou astro\-nomií se zabývají skripta:
\begin{enumerate}
    \item Geodetická astronomie 10: Toto skriptum bylo vydáno nakladatelstvím ČVUT a zabývá se obecně geodetickou astronomií. Ve skriptu je popsána nauka o~čase, metody geodetické astronomie, přístroje pro astronomické měření včetně přístrojů pro měření času. Skriptum dále obsahuje rozbory přesnosti pro astronomické měření a zmiňuje i požadavky na přesnost určení času \cite{kostelecky_geodeticka_astronomie}.
    \item Geodetická astronomie I. a základy kosmické geodézie: Toto skriptum, bylo \text{vydáno} nakladatelstvím VUTIUM a zabývá se geodetickou astronomií a \text{kosmickou} geodezií. Pro tuto práci je v~něm důležitá především kapitola o~časech a časových systémech \cite{fixel_geodeticka_astronomie}.
\end{enumerate}
Tvorbou zařízení, které umí synchronizovat svůj vnitřní čas se zabývají práce:
\begin{enumerate}
    \item „Časový normál pro centralizovanou správu elektronických hodin“: Tato bakalářská práce se zabývá návrhem a realizací hodinové ústředny, která po zisku časového normálu vzdáleně řídí elektronické hodiny. K~synchronizaci času tato práce využívá tři nezávislé zdroje (DCF77, GPS a NTP server) \cite{pilik_casovy_normal}.
    \item „Elektronické hodiny“: Tato bakalářská práce se zabývá synchronizací času \text{pomocí} DCF77 a GPS přijímače \cite{havlicek_elektronicke_hodiny}.
    \item „Synchronizace času pro odloučená měřící stanoviště“: Tato diplomová práce se zabývá tvorbou zařízení, které pomocí signálu DCF77 synchronizuje čas osobního počítače \cite{zeis_synchronizace_casu}.%Tato práce se zabývá
\end{enumerate}