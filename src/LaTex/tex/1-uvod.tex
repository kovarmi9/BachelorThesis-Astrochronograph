\chapter{Úvod}
\label{1-uvod}
V~minulosti bylo velmi obtížné určit přesnou polohu objektů na Zemi. Poloha objektů na Zemi byla určována především pomocí geodetické astronomie. Určovaly se astronomické zeměpisné souřadnice (astronomická zeměpisná šířka a astronomická zeměpisná délka) a astronomické azimuty. Během astronomického měření se na rozdíl od klasické geodézie měří na pohyblivé cíle a je tedy nutné zaznamenávat i čas jednotlivých měření. Přesnost určení času má významný vliv na přesnost konkrétních výsledků. 

V~dnešní době byly metody geodetické astronomie z~velké části nahrazeny metodami kosmické geodezie, jako jsou \zk{GNSS} a VLBI. Avšak pro výukové účely má \text{astronomické} měření stále svůj význam. V~současné době se v~rámci výuky \text{teoretické} geodezie zaměřují astronomické azimuty na Slunce. Během tohoto měření se synchro\-nizují stopky (zpravidla na chytrém telefonu) pomocí analogového přijímače signálu DCF77 ručně, a při vlastním měření se záznam provádí druhou osobou na slovní pokyn měřiče. To je příčinou velkých nepřesností v~určení času, které \text{negativně} ovlivňují výsledky celého astronomického měření.

Cílem této práce je navrhnout a vytvořit zařízení schopné přijímat a \text{zaznamenávat} koordinovaný čas pro účely astronomického měření (Astrochronograf). Toto zařízení by mělo být schopno automaticky synchronizovat svůj vnitřní čas, aby \text{nedocházelo} k~chybě z~ruční \text{synchronizace}. Mělo by být jednoduché na ovládání tak, aby mohl astronomické měření snadno provádět i jednotlivec, díky čemuž se eliminuje \text{prodleva} způsobená slovním pokynem. Zařízení by mělo být také schopné \text{informovat} měřiče o~průběhu měření, včetně stavu synchronizace a posledního změřeného času, a umožnit snadný export přehledně zaznamenaných měřených dat.

Výsledné zařízení by mělo být testováno, aby byla ověřena jeho funkčnost. Dále by mělo být zjištěno, jak přesně je zařízení schopno synchronizovat svůj vnitřní čas, jak přesně je schopné zaznamenat měřený čas a jaký vliv má zlepšení určení času na konečné výsledky astronomického měření.